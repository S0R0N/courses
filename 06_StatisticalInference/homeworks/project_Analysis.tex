\documentclass[]{article}
\usepackage{lmodern}
\usepackage{amssymb,amsmath}
\usepackage{ifxetex,ifluatex}
\usepackage{fixltx2e} % provides \textsubscript
\ifnum 0\ifxetex 1\fi\ifluatex 1\fi=0 % if pdftex
  \usepackage[T1]{fontenc}
  \usepackage[utf8]{inputenc}
\else % if luatex or xelatex
  \ifxetex
    \usepackage{mathspec}
  \else
    \usepackage{fontspec}
  \fi
  \defaultfontfeatures{Ligatures=TeX,Scale=MatchLowercase}
\fi
% use upquote if available, for straight quotes in verbatim environments
\IfFileExists{upquote.sty}{\usepackage{upquote}}{}
% use microtype if available
\IfFileExists{microtype.sty}{%
\usepackage{microtype}
\UseMicrotypeSet[protrusion]{basicmath} % disable protrusion for tt fonts
}{}
\usepackage[margin=1in]{geometry}
\usepackage{hyperref}
\hypersetup{unicode=true,
            pdftitle={Peer-graded Assignment: Statistical Inference Course Project - part 2},
            pdfauthor={Santiago Ruiz Navas},
            pdfborder={0 0 0},
            breaklinks=true}
\urlstyle{same}  % don't use monospace font for urls
\usepackage{color}
\usepackage{fancyvrb}
\newcommand{\VerbBar}{|}
\newcommand{\VERB}{\Verb[commandchars=\\\{\}]}
\DefineVerbatimEnvironment{Highlighting}{Verbatim}{commandchars=\\\{\}}
% Add ',fontsize=\small' for more characters per line
\usepackage{framed}
\definecolor{shadecolor}{RGB}{248,248,248}
\newenvironment{Shaded}{\begin{snugshade}}{\end{snugshade}}
\newcommand{\AlertTok}[1]{\textcolor[rgb]{0.94,0.16,0.16}{#1}}
\newcommand{\AnnotationTok}[1]{\textcolor[rgb]{0.56,0.35,0.01}{\textbf{\textit{#1}}}}
\newcommand{\AttributeTok}[1]{\textcolor[rgb]{0.77,0.63,0.00}{#1}}
\newcommand{\BaseNTok}[1]{\textcolor[rgb]{0.00,0.00,0.81}{#1}}
\newcommand{\BuiltInTok}[1]{#1}
\newcommand{\CharTok}[1]{\textcolor[rgb]{0.31,0.60,0.02}{#1}}
\newcommand{\CommentTok}[1]{\textcolor[rgb]{0.56,0.35,0.01}{\textit{#1}}}
\newcommand{\CommentVarTok}[1]{\textcolor[rgb]{0.56,0.35,0.01}{\textbf{\textit{#1}}}}
\newcommand{\ConstantTok}[1]{\textcolor[rgb]{0.00,0.00,0.00}{#1}}
\newcommand{\ControlFlowTok}[1]{\textcolor[rgb]{0.13,0.29,0.53}{\textbf{#1}}}
\newcommand{\DataTypeTok}[1]{\textcolor[rgb]{0.13,0.29,0.53}{#1}}
\newcommand{\DecValTok}[1]{\textcolor[rgb]{0.00,0.00,0.81}{#1}}
\newcommand{\DocumentationTok}[1]{\textcolor[rgb]{0.56,0.35,0.01}{\textbf{\textit{#1}}}}
\newcommand{\ErrorTok}[1]{\textcolor[rgb]{0.64,0.00,0.00}{\textbf{#1}}}
\newcommand{\ExtensionTok}[1]{#1}
\newcommand{\FloatTok}[1]{\textcolor[rgb]{0.00,0.00,0.81}{#1}}
\newcommand{\FunctionTok}[1]{\textcolor[rgb]{0.00,0.00,0.00}{#1}}
\newcommand{\ImportTok}[1]{#1}
\newcommand{\InformationTok}[1]{\textcolor[rgb]{0.56,0.35,0.01}{\textbf{\textit{#1}}}}
\newcommand{\KeywordTok}[1]{\textcolor[rgb]{0.13,0.29,0.53}{\textbf{#1}}}
\newcommand{\NormalTok}[1]{#1}
\newcommand{\OperatorTok}[1]{\textcolor[rgb]{0.81,0.36,0.00}{\textbf{#1}}}
\newcommand{\OtherTok}[1]{\textcolor[rgb]{0.56,0.35,0.01}{#1}}
\newcommand{\PreprocessorTok}[1]{\textcolor[rgb]{0.56,0.35,0.01}{\textit{#1}}}
\newcommand{\RegionMarkerTok}[1]{#1}
\newcommand{\SpecialCharTok}[1]{\textcolor[rgb]{0.00,0.00,0.00}{#1}}
\newcommand{\SpecialStringTok}[1]{\textcolor[rgb]{0.31,0.60,0.02}{#1}}
\newcommand{\StringTok}[1]{\textcolor[rgb]{0.31,0.60,0.02}{#1}}
\newcommand{\VariableTok}[1]{\textcolor[rgb]{0.00,0.00,0.00}{#1}}
\newcommand{\VerbatimStringTok}[1]{\textcolor[rgb]{0.31,0.60,0.02}{#1}}
\newcommand{\WarningTok}[1]{\textcolor[rgb]{0.56,0.35,0.01}{\textbf{\textit{#1}}}}
\usepackage{graphicx,grffile}
\makeatletter
\def\maxwidth{\ifdim\Gin@nat@width>\linewidth\linewidth\else\Gin@nat@width\fi}
\def\maxheight{\ifdim\Gin@nat@height>\textheight\textheight\else\Gin@nat@height\fi}
\makeatother
% Scale images if necessary, so that they will not overflow the page
% margins by default, and it is still possible to overwrite the defaults
% using explicit options in \includegraphics[width, height, ...]{}
\setkeys{Gin}{width=\maxwidth,height=\maxheight,keepaspectratio}
\IfFileExists{parskip.sty}{%
\usepackage{parskip}
}{% else
\setlength{\parindent}{0pt}
\setlength{\parskip}{6pt plus 2pt minus 1pt}
}
\setlength{\emergencystretch}{3em}  % prevent overfull lines
\providecommand{\tightlist}{%
  \setlength{\itemsep}{0pt}\setlength{\parskip}{0pt}}
\setcounter{secnumdepth}{0}
% Redefines (sub)paragraphs to behave more like sections
\ifx\paragraph\undefined\else
\let\oldparagraph\paragraph
\renewcommand{\paragraph}[1]{\oldparagraph{#1}\mbox{}}
\fi
\ifx\subparagraph\undefined\else
\let\oldsubparagraph\subparagraph
\renewcommand{\subparagraph}[1]{\oldsubparagraph{#1}\mbox{}}
\fi

%%% Use protect on footnotes to avoid problems with footnotes in titles
\let\rmarkdownfootnote\footnote%
\def\footnote{\protect\rmarkdownfootnote}

%%% Change title format to be more compact
\usepackage{titling}

% Create subtitle command for use in maketitle
\providecommand{\subtitle}[1]{
  \posttitle{
    \begin{center}\large#1\end{center}
    }
}

\setlength{\droptitle}{-2em}

  \title{Peer-graded Assignment: Statistical Inference Course Project - part 2}
    \pretitle{\vspace{\droptitle}\centering\huge}
  \posttitle{\par}
    \author{Santiago Ruiz Navas}
    \preauthor{\centering\large\emph}
  \postauthor{\par}
      \predate{\centering\large\emph}
  \postdate{\par}
    \date{11/12/2019}


\begin{document}
\maketitle

\hypertarget{overview}{%
\section{OVERVIEW}\label{overview}}

This document consits of the analysis required for the secondpart of the
Statistical Inference Course Project.

\hypertarget{basic-inferential-data-analysis-instructions}{%
\section{2 Basic Inferential Data Analysis
Instructions}\label{basic-inferential-data-analysis-instructions}}

\hypertarget{summary-of-the-data}{%
\subsection{2.1 Summary of the data}\label{summary-of-the-data}}

Running the summary command the basic statistics and characteristics of
the dataset are presented bellow

\begin{Shaded}
\begin{Highlighting}[]
\KeywordTok{summary}\NormalTok{(ToothGrowth)}
\end{Highlighting}
\end{Shaded}

\begin{verbatim}
##       len        supp         dose      
##  Min.   : 4.20   OJ:30   Min.   :0.500  
##  1st Qu.:13.07   VC:30   1st Qu.:0.500  
##  Median :19.25           Median :1.000  
##  Mean   :18.81           Mean   :1.167  
##  3rd Qu.:25.27           3rd Qu.:2.000  
##  Max.   :33.90           Max.   :2.000
\end{verbatim}

\hypertarget{confidence-intervals-for-suup-and-doses}{%
\subsection{2.2 confidence intervals for suup and
doses}\label{confidence-intervals-for-suup-and-doses}}

Did the student perform some relevant confidence intervals and/or tests?
\#\#\# 2.3.1 Confidence interval for suup Test for supp VC - OJ

\begin{Shaded}
\begin{Highlighting}[]
\KeywordTok{t.test}\NormalTok{(data.VC}\OperatorTok{$}\NormalTok{len, data.OJ}\OperatorTok{$}\NormalTok{len, }\DataTypeTok{paired =} \OtherTok{FALSE}\NormalTok{, }\DataTypeTok{var.equal =} \OtherTok{FALSE}\NormalTok{)}
\end{Highlighting}
\end{Shaded}

\begin{verbatim}
## 
##  Welch Two Sample t-test
## 
## data:  data.VC$len and data.OJ$len
## t = -1.9153, df = 55.309, p-value = 0.06063
## alternative hypothesis: true difference in means is not equal to 0
## 95 percent confidence interval:
##  -7.5710156  0.1710156
## sample estimates:
## mean of x mean of y 
##  16.96333  20.66333
\end{verbatim}

\hypertarget{interpretation-of-the-t-test-for-supp}{%
\subsubsection{Interpretation of the T-test for
supp}\label{interpretation-of-the-t-test-for-supp}}

For paired the two sided t-test at 95\% provides an interval -7.57 and
0.17, the interval includes cero. Therefore, when comparing OJ and VC we
fail to reject the null hipothesis and we can not conclude wether OJ or
VC have an effect on tooth lenght at 95\% of confidence inteval.

\hypertarget{confidence-interval-for-dose}{%
\subsubsection{2.2.2 Confidence interval for
dose}\label{confidence-interval-for-dose}}

\hypertarget{dose-1-0.5---dose-2-1}{%
\subsubsection{dose 1 (0.5) - dose 2 (1)}\label{dose-1-0.5---dose-2-1}}

\begin{Shaded}
\begin{Highlighting}[]
\NormalTok{data12 <-}\StringTok{ }\KeywordTok{t.test}\NormalTok{(df12, }\DataTypeTok{paired =} \OtherTok{FALSE}\NormalTok{, }\DataTypeTok{var.equal =} \OtherTok{FALSE}\NormalTok{)}
\end{Highlighting}
\end{Shaded}

\hypertarget{dose-1-0.5---dose-3-2}{%
\subsubsection{2.2.3 dose 1 (0.5) - dose 3
(2)}\label{dose-1-0.5---dose-3-2}}

\begin{Shaded}
\begin{Highlighting}[]
\NormalTok{data13<-}\StringTok{ }\KeywordTok{t.test}\NormalTok{(df13, }\DataTypeTok{paired =} \OtherTok{FALSE}\NormalTok{, }\DataTypeTok{var.equal =} \OtherTok{FALSE}\NormalTok{)}
\end{Highlighting}
\end{Shaded}

\hypertarget{dose-2-1---dose-3-2}{%
\subsubsection{2.2.4 dose 2 (1) - dose 3
(2)}\label{dose-2-1---dose-3-2}}

\begin{Shaded}
\begin{Highlighting}[]
\NormalTok{data23 <-}\StringTok{ }\KeywordTok{t.test}\NormalTok{(df23, }\DataTypeTok{paired =} \OtherTok{FALSE}\NormalTok{, }\DataTypeTok{var.equal =} \OtherTok{FALSE}\NormalTok{)}
\end{Highlighting}
\end{Shaded}

\hypertarget{interpretation-of-the-t-test-for-dose}{%
\subsubsection{Interpretation of the t-test for
dose}\label{interpretation-of-the-t-test-for-dose}}

Two main conclusions are obtaned after looking into the confidence
intervals of the differences between the doses. First, in a 95\% double
sided confidence interval, when the dose is bigger the difference in
tooth lenght is bigger. Second, that with the available data it is not
possible to tell if OJ and VC have an effect on tooth lenght.

\hypertarget{assumptions-for-this-analysis}{%
\subsubsection{Assumptions for this
analysis}\label{assumptions-for-this-analysis}}

\begin{itemize}
\tightlist
\item
  The data in the toothgrowth dataset is
\item
  not paired
\item
  independent
\item
  the tooth length reported for each of the doses and the type of supps
  have different variances
\item
  the tooth length in the dataset follows a t-student distribution
\end{itemize}


\end{document}
